\documentclass[a4paper]{article}

%% Language and font encodings
\usepackage[german]{babel}
\usepackage[utf8x]{inputenc}
\usepackage[T1]{fontenc}
\usepackage{float}

%% Sets page size and margins
\usepackage[a4paper,top=3cm,bottom=2cm,left=3cm,right=3cm,marginparwidth=1.75cm]{geometry}

%% Useful packages
\usepackage{amsmath}
\usepackage{graphicx}
\usepackage[colorinlistoftodos]{todonotes}
\usepackage[colorlinks=true, allcolors=blue]{hyperref}

%% Definitions
\newcommand{\entspricht}{\mathrel{\widehat{=}}}

\title{Open CheatSheet: GdI-Edition}
\author{OpenSource \url{http://de.wikipedia.org/}Code Here}

\begin{document}
\maketitle

\section{Tabellenwerk}
\begin{table}[H]
\centering
\caption{Umrechnung zu dezimal bis $b^{12}$}
\label{gaengigeWerte}
\begin{tabular}{l|c|c|c|c|c|c|c|c|c|c|c|c|c}
         & $b^{0}$&$b^{1}$&$b^{2}$&$b^{3}$&$b^{4}$&$b^{5}$&$b^{6}$&$b^{7}$&$b^{8}$&$b^{9}$&$b^{10}$&$b^{11}$&$b^{12}$ \\ \hline\hline
$b = 2$  & 1      & 2     & 4     & 8     & 16    & 32    & 64    & 128   & 256   & 512   & 1024   & 2048   & 4096    \\
$b = 8$  & 1      & 8     & 64    & 512   & 4096  & -     & -     & -     & -     & -     & -      & -      & -       \\
$b = 16$ & 1      & 16    & 256   & 4096  & 65536 & -     & -     & -     & -     & -     & -      & -      & -       \\
\end{tabular}
\end{table}

\begin{table}[H]
\centering
\caption{Gängige Zahlensysteme: Darstellungen bis Wert 15}
\label{hexWerte}
\begin{tabular}{l|c|c|c|c|c|c|c|c|c|c|c|c|c|c|c|c}
$dez$ & 0 & 1 & 2  & 3  & 4   & 5   & 6     & 7     & 8    & 9    & 10   & 11   & 12   & 13   & 14   & 15   \\ \hline\hline
$bin$ & 0 & 1 & 10 & 11 & 100 & 101 & 110   & 111   & 1000 & 1001 & 1010 & 1011 & 1100 & 1101 & 1110 & 1111 \\
$oct$ & 0 & 1 & 2  & 3  & 4   & 5   & 6     & 7     & 10   & 11   & 12   & 13   & 14   & 15   & 16   & 17   \\
$hex$ & 0 & 1 & 2  & 3  & 4   & 5   & 6     & 7     & 8    & 9    & A    & B    & C    & D    & E    & F    \\
\end{tabular}
\end{table}

\begin{table}[H]
\centering
\caption{vielfache von $10_{10}$ (Nützlich für die Berechnung von Dezimalzahlen im Quellsystem)}
\label{vielfacheVon10}
\begin{tabular}{cc|c|c|c|c|c|c|c|c}
Basis &10    & 2    & 3   & 4   & 5   & 6   & 7   & 8   & 9  \\\hline
      &10    & 1010 & 101 & 22  & 20  & 14  & 13  & 12  & 11 \\\hline
      &20    &      & 202 & 110 & 40  & 32  & 26  & 24  & 22 \\\hline
      &30    &      &     & 132 & 110 & 50  & 42  & 36  & 33 \\\hline
      &40    &      &     &     & 130 & 104 & 55  & 50  & 44 \\\hline
      &50    &      &     &     &     & 122 & 101 & 62  & 55 \\\hline
      &60    &      &     &     &     &     & 114 & 74  & 66 \\\hline
      &70    &      &     &     &     &     &     & 106 & 77 \\\hline
      &80    &      &     &     &     &     &     &     & 88
\end{tabular}
\end{table}

\section{Umrechnen}
\subsection{Im Dezimalsystem}
\subsubsection{\texorpdfstring{$b \neq 10$ zu $b = 10$ (beliebige Basis ungleich 10 zu Basis 10)}{b-adisch ungleich 10 zu Basis 10}}
Stellen addieren nach folgender Vorschrift:
\[
 \sum \limits_{n=0}^{N-1} (a_n * b^{n}), a_n \in \{0, \ldots , b-1\}, i \in \{0, \ldots ,N-1\}
\]
wobei $N$: Anzahl der Stellen, $n$: Stelle der Zahl und $a_n$: Wert der Ziffer an Stelle $n$.

\subsubsection{\texorpdfstring{Ein Beispiel: $(124032)_5$}{Ein Beispiel: Basis 5}}
\[
(124032)_5 = (1*5^{5})+(2*5^{4})+(4*5^{3})+(0*5^{2})+(3*5^{1})+(2*5^{0}) = 3125 + 1250 + 500 + 0 + 15 + 2 = 4892
\]

\subsection{\texorpdfstring{Umrechnen im Quellsystem: Von $b \neq 10$ zu $b = 10$}{Rechnen im Quellsystem: Von b-adisch ungleich 10 zu Basis 10}}
Rechnen im Quellsystem zur Umwandlung von Zahlen beliebiger b-adischer Systeme zum dezimal System

\subsubsection{Der Algorithmus in Worten}
\begin{enumerate}
\item Teile die umzurechnende Zahl durch die Repräsentation der $(10)_{10}$ im Quellsystem bis das Ergebnis der Subtraktionen (innerhalb des Divisionsalgorithmus) nicht mehr durch "10" teilbar sind. Das letzte Ergebnis der Subtraktion ist der Divisionsrest.
\item Wiederhole Schritt 1 für jedes Ergebnis (ohne Rest) bis das Resultat nicht mehr durch die Repräsentation von $(10)_{10}$ teilbar ist.
\item Die Reste der Divisionen sind die Quellsystemrepräsentanten der einzelnen Stellen der zu berechnenden Dezimalzahl. Die hochwertigste Stelle entspricht der zuletzt vorgenommenen Division.
\end{enumerate}

\subsubsection{\texorpdfstring{Ein Beispiel: $(120102)_3$ zu $(416)_{10}$}{Ein Beispiel: 120102 Basis 3 zu Dezimal 416}}
Folgende Tabelle hilft bei der Berechnung der Divisionen. Man braucht nur vielfache bis $b-1$ zum berechnen der Divisionen.
\begin{table}[H]
\centering
\caption{Vielfache von $(10)_{10}$ im 3-er System}
\label{vielfacheVon10in3}
\begin{tabular}{c|c}
Basis 10 & basis 3 \\\hline
10       & 101     \\\hline
20       & 202
\end{tabular}
\end{table}
\begin{eqnarray}
120102 / 101 =  1112\;R\:20 \\
1112 / 101 = 11\;R\:1 \\
11 / 101 = 0\;R\:11
\end{eqnarray}
Aus (3) $ \Rightarrow  (11)_3 \entspricht (4)_{10} $ \\
Aus (2) $ \Rightarrow  (1)_3 \entspricht (1)_{10} $ \\
Aus (1) $ \Rightarrow  (20)_3 \entspricht (6)_{10} $ \\

$ \Rightarrow (416)_{10} $

\section{Vorzeichenbehaftete Darstellungen}
\subsection{kleinste / größte darstellbare Zahl mit N Bits}
\paragraph{Exzess-N}
kleinste: -N; Größte: N-1
\paragraph{b-1-Komplement}
$[ -(2^{N-1}-1) … +(2^{N-1}-1)]$\\
doppelte null: 0000 und 1111 sind beides null
\paragraph{b-Komplement}
$[ - 2^{N-1} … +2^{N-1} - 1]$

\subsection{Rechnen mit dem einer Komplement}
Den einer Rücklauf beachten (carry bit von höchster stelle wird (falls vorhanden) zum Ergebnis addiert (Ergebnis + 1)
\subsection{Rechnen mit dem zweier Komplement}
nach der Komplementbildung eins addieren.
\subsection{Rechnen im Exzesscode}

\subsection{Welche Zahlen sind in einem b-adischen System negativ}
\paragraph{bei geraden basen} ist die erste Ziffer größergleich $ b / 2 $ gilt die zahl als negativ ansonsten ist sie positiv \\
So sind die vorzeichenlosen Zahlen $1 … (b^N/2)-1$ im b-Komplement positiv, $(b^N/2) … b^N-1$ sind negativ, $0$ ist null.\\
Im (b-1)-Komplement gilt, dass die Zahlen $(b^N/2) … b^N-2$ negativ sind, $b^N-1$ ist minus null.
\paragraph{bei ungeraden basen}
Hier wird die Trennlinie bei der Zahl $(aaa...a)_b$ gezogen, wobei $a=(b-1)/2$ ist. Bei$ b=3$ ist es die Zahl $(111...1)_3$.
\section{Reele Zahlen}
\subsection{2-adische Entwicklung (Dezimal zu Binär wandeln)}
\begin{enumerate} 
\item Teile durch 2
\item Erstes Ergebnis: Erste Stelle nach dem Komma 
\item Wenn Ergebnis < 0 beudetet dies eine 0 für diese Nachkommastelle
\end{enumerate}
Trick: Wenn die Zahl welche binär dargestellt werden soll eine Zweierpotenz im Nenner hat, dann stellt der Wert des Exponenten die Anzahl der Nachkommastellen dar. Also bspw. $ 1/8 = 1/2^3 $ dar. Daraus folgt 3 Nachkommastellen (weil der Exponent die drei ist). Die Zahl auf dem Bruch wird dann von rechts eingeschoben also bspw. $2/8 = 2/2^3 $ => drei Nachkommastellen und $ 2 = 01 $ daraus folgt das Ergebnis: $ 0.010 = 1/4$


\subsection{Konvertieren zwischen Binär/Hexadezimal/Dezimal}
Folgende Beispiele beschreiben das Schema:

\paragraph{Vorkomma Anteil} (siehe Kapitel 2 und 3 - Umrechnen ganzer Zahlen)
\begin{eqnarray}
10110.110010 \\
0001 = (1)_{16} \\
0110 = (6)_{16}\\
\Rightarrow (16)_{16} \\
(16)_{16} = (22)_{10}
\end{eqnarray}

\paragraph{Nachkomma Anteil zu dez}
Aufsummieren:
$N*b^{-1}+N*b^{-2}+...+N*b^{-i} $ wobei i die letzte Nachkommastelle ist und N der wert der stelle

\paragraph{Shortcut bei Zahlen mit nur einsen nach dem komma}
Wert von $0.1111$ (also vier stellen nach dem Komma): $(1-2^{-4}) = 2^{-1}+2^{-2}+2^{-3}+2^{-4} = \frac{15}{16}$

\paragraph{Nachkommaanteil von bin zu hex}
Vierergrüppchen bilden: $ \Rightarrow (10110.110010)_{2} = (0001)(0110).(1100)(1000) = (16.C8)_{16}$ 

\subsection{Gleitkommazahlen}
\paragraph{normalisiert} ist eine Gleitkommazahl wenn die Ziffern nach dem Komma sich nicht mehr weiter nach links verschieben lassen ohne das werte über das Komma "hinausrutschen". Die folgenden Werte sind bereits normalisiert:
\begin{eqnarray}
0.11001*2^{-39} \\
0.001011*8^4
\end{eqnarray}
\paragraph{Unterlauf- und Überlaufbereich}
Unterlauf: Kleinste zulässige(!) Darstellung wählen und Dezimalwert berechnen. Dabei darauf achten: wenn normalisierung gefordert wird, ist die kleinste zulässige mantisse nicht 0.000001 sonder 0.1. In IEEE ist bspw. keine normalisierung gefordert, daher kann man hier wirklich den kleinst möglichen mantissenwert annehmen.

\subsubsection{IEEE-Single-Precision Floating Point}

\paragraph{Umrechnung IEEE zu Dez} von links: Bit 1: Vorzeichen, Bits 2-9 (8bit): Exponent, Bits 10-32 (23bit): Mantisse.\\
{\Large $z = \pm1.m_1 ... m_{23}*2^{e_1 ... e_8-127}$ mit VZ $1 \stackrel{\wedge}{=} - $ und $0 \stackrel{\wedge}{=} +$} 
\\ 






\end{document}


